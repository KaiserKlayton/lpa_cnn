\documentclass[]{article}
\usepackage{lmodern}
\usepackage{amssymb,amsmath}
\usepackage{ifxetex,ifluatex}
\usepackage{fixltx2e} % provides \textsubscript
\ifnum 0\ifxetex 1\fi\ifluatex 1\fi=0 % if pdftex
  \usepackage[T1]{fontenc}
  \usepackage[utf8]{inputenc}
\else % if luatex or xelatex
  \ifxetex
    \usepackage{mathspec}
  \else
    \usepackage{fontspec}
  \fi
  \defaultfontfeatures{Ligatures=TeX,Scale=MatchLowercase}
\fi
% use upquote if available, for straight quotes in verbatim environments
\IfFileExists{upquote.sty}{\usepackage{upquote}}{}
% use microtype if available
\IfFileExists{microtype.sty}{%
\usepackage{microtype}
\UseMicrotypeSet[protrusion]{basicmath} % disable protrusion for tt fonts
}{}
\usepackage{hyperref}
\hypersetup{unicode=true,
            pdfborder={0 0 0},
            breaklinks=true}
\urlstyle{same}  % don't use monospace font for urls
\IfFileExists{parskip.sty}{%
\usepackage{parskip}
}{% else
\setlength{\parindent}{0pt}
\setlength{\parskip}{6pt plus 2pt minus 1pt}
}
\setlength{\emergencystretch}{3em}  % prevent overfull lines
\providecommand{\tightlist}{%
  \setlength{\itemsep}{0pt}\setlength{\parskip}{0pt}}
\setcounter{secnumdepth}{0}
% Redefines (sub)paragraphs to behave more like sections
\ifx\paragraph\undefined\else
\let\oldparagraph\paragraph
\renewcommand{\paragraph}[1]{\oldparagraph{#1}\mbox{}}
\fi
\ifx\subparagraph\undefined\else
\let\oldsubparagraph\subparagraph
\renewcommand{\subparagraph}[1]{\oldsubparagraph{#1}\mbox{}}
\fi

\date{}

\begin{document}

\section{Obtaining the system}\label{Obtaining the system}

Clone the repository from GitHub: \url{git@github.com:KaiserKlayton/lpa\_cnn.git}

or download as a zip @ \url{https://github.com/KaiserKlayton/lpa_cnn/archive/master.zip}

\section{Dependencies}\label{dependencies}

gcc 5.4 w/ Eigen 3 \& Armadillo

Python 2.7 w/ NumPy \& PIL

R w/ gtools

\section{Setup}\label{setup}

Add caffe Python module directory to \texttt{\$PYTHONPATH}.

Have the following files in place for each desired model:

\begin{verbatim}
  models/<model_name>/<model_name.caffemodel>
  models/<model_name>/<model_name.prototxt>
  ...
\end{verbatim}

Have the following input file in place for each installed model:

\begin{verbatim}
  inputs/<model_name>/production/<input_file_name.csv>
\end{verbatim}

having the form:

\begin{verbatim}
  <img_0_label><img_0_channel_1>...<img_0_channel_2><img_0_channel_3>
  <img_1_label><img_1_channel_1>...<img_1_channel_2><img_1_channel_3>
  ...
\end{verbatim}

\section{Reproduction}\label{reproduction}

To reproduce the experiments with the installed models, call
\texttt{run/run\_routine.py}.

Results are written to \texttt{results/}.

\section{Installing new
models}\label{installing-new-models}

A great resource for finding new Caffe models is Model Zoo @
\url{https://github.com/BVLC/caffe/wiki/Model-Zoo}

To install a new model, follow the \texttt{Setup} directions above,
providing an appropriate and consistent model name as
\texttt{\textless{}model\_name\textgreater{}}.

\texttt{NOTE} that when preparing .prototxt files, \texttt{lpa\_cnn}
supports the following parameters:

\begin{verbatim}
  layer_types = ['convolution', 'pooling', 'relu', 'eltwise', 'innerproduct']
  param_types = ['num_output', 'pad', 'kernel_size', 'stride', 'bias_term', 'pool']
  special_types = ['shape', 'input_dim']
  shape_dims = ['n','d','w','h']
\end{verbatim}

\texttt{NOTE} that batch processing is not supported.

\end{document}
